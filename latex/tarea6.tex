\documentclass{article} % Clase del documento
\usepackage{enumitem}
\usepackage{graphicx}
\usepackage{listings}

\title{\textbf{Introducci\'on a Ciencias de la Computaci\'on (I)}}
\author{\textbf{P\'erez Servin Darshan Israel}}
\date{\textbf{Tarea 6: Herencia\\ Fecha de entrega: Mi\'ercoles 22 de Noviembre del 2023}} % Fecha actual

\begin{document}
\maketitle % Crea el título con la información proporcionada anteriormente

\begin{enumerate}
\item \textbf{Explica qué significa que un atributo o método tenga visibilidad \texttt{protected}.}\\\\
    Un atrubuto o método que sea protected puede ser visible solo dentro de la clase
    y en sus clases heredadas, a comparación del privado que solo  es visible dentro 
    de la clase padre.\\
    Por ejemplo si se crea un método o atribúto en una clase y de esta se hereda, este
    mismo puede ser usado en la clase heredada sin necesidad de alterar lo que se encuentra
    en la clase ``padre".
\item \textbf{Indica la diferencia entre la sobrecarga de un método y la redefinición de un método.}
    \begin{enumerate}[label={}]
        \item \textbf{Sobrecarga de Método:}\\
        Una sobrecarga de método Es tener varios métodos con el mismo nombre en una sola clase, 
        aunque sean diferentes en lo que hagan.
        \item \textbf{Redefinición de Método}\\
        Una redefinición es cuando se crea un método en una clase padre y se hereda en otra clase
        bajo el mismo nombre implementando nuevas funciones sobre el anterior.
    \end{enumerate}

\item \textbf{Indica cuáles atributos y métodos de una clase pueden ser heredados a otra.}\\
    Los atributos y métodos que se pueden heredar de una clase a otra son aquellos que se 
    encuentran en \texttt{protected} y en \texttt{public}. No se pueden heredar aquellos que se
    encuentren como \texttt{final} o en \texttt{static}.
\item \textbf{Menciona de qué manera se relaciona la redefinición de métodos con el polimorfismo.}\\
    La redefinición de métodos, ocurre cuando una clase hija proporciona una implementación 
    nueva y específica para un método que ya está definido en su clase padre. Usualmente
    para evitar errores en el compilador se opta por usar la sintaxis ``\texttt{@Override}" antes
    del método. Esto se lleva de la mano con el polimorfismo, ya que este se refiere cuando
    un método que se redefine con una firma idéntica o compatible a su método de la clase base, de
    manera de que se puede invocar de dos formas diferentes. Como por ejemplo estos dos objetos
    serían distintos y por ende si se invoca con el mismo método, actuaría segun que hace el método
    de la clase:\\
    \begin{lstlisting}[language=Java, title={CÓDIGO EJEMPLO:}]
        ClasePadre objeto1 = new ClasePadre();
        ClasePadre objeto1 = new ClaseHija();
        
        objeto1.metodo();
        //Hace lo que se encuentre en la clase padre
        objeto2.metodo();
        //Hace lo que se encuentre en la clase hija

    \end{lstlisting}
    
\item \textbf{Menciona una diferencia y una similitud entre clases abstractas e interfaces.}\\\\
    \textbf{Diferencias:}\\
    En una clase abstracta se puede almacenar métodos ya sean abstractos o concretos.\\
    En una intefaz SOLO se pueden almacenar métodos abstractos y variables finales.\\\\
    \textbf{Similitud:}\\
    Ambas se usan para de alguna manera "heredar" métodos que no estan definidos aún, una
    buena guía para saber que métodos se pueden usar al heredarse.
\end{enumerate}

\end{document}
